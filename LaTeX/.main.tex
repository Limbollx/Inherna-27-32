\documentclass{article}
\usepackage[french]{babel}
\usepackage[utf8]{inputenc}
\usepackage[T1]{fontenc}

\begin{document}

Les marées correspondent aux variations périodiques du niveau des mers et des océans, sous l'influence des forces gravitationnelles exercées par la Lune et le Soleil sur la Terre. Si leur manifestation en bord de côte est bien connue du grand public, le phénomène qui les gouverne est en réalité le fruit d'interactions complexes entre les corps célestes, la rotation terrestre, et la géométrie des bassins océaniques.

Comprendre et prédire l'évolution des marées suppose de s'appuyer sur des modèles physiques et mathématiques capables de traduire ces interactions en grandeurs mesurables. Dans ce travail, nous cherchons à représenter de manière simplifiée cette dynamique, en étudiant comment le niveau de l'eau varie au cours du temps dans un cadre réduit mais cohérent avec les phénomènes réels.

\end{document}